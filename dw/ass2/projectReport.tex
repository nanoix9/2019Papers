% Options for packages loaded elsewhere
\PassOptionsToPackage{unicode}{hyperref}
\PassOptionsToPackage{hyphens}{url}
%
\documentclass[
  12pt,
  a4paper,
]{article}

\usepackage{etoolbox}
% \setmainfont{Linux Libertine O}
\AtBeginEnvironment{quote}{\it\small}

\usepackage[toc,page,header]{appendix}
% \renewcommand{\appendixpagename}{\centering Appendices111}

% \usepackage[colorlinks=true, linkcolor=blue, urlcolor=blue, citecolor=blue, anchorcolor=blue]{hyperref}

\usepackage[]{mathpazo}
\usepackage{setspace}
% \usepackage{lmodern,bm}
% \usepackage{unicode-math}
\usepackage{amssymb}
\usepackage{amsmath}
% \usepackage{amsfonts}
\usepackage{ifxetex,ifluatex}
\ifnum 0\ifxetex 1\fi\ifluatex 1\fi=0 % if pdftex
  \usepackage[T1]{fontenc}
  \usepackage[utf8]{inputenc}
  % \usepackage{textcomp} % provide euro and other symbols
  % \usepackage{newtxtext,newtxmath}
  % \usepackage{mathptmx} \usepackage{tgtermes}
  % \usepackage{FiraSans} 
  % \usepackage[sfdefault]{FiraSans} \usepackage{FiraMono} \renewcommand*\familydefault{\sfdefault}
  % \usepackage{tgpagella}
  % \usepackage[utopia]{mathdesign}
  % \usepackage[math]{iwona}
  % \usepackage[sfdefault,scaled=.85]{FiraSans} \usepackage{newtxsf}
  % \usepackage{cmbright}
  % \usepackage{ccfonts,eulervm}  \usepackage[T1]{fontenc}
  % \usepackage[math]{kurier}
  % \usepackage{fourier}
  % \usepackage{mathpazo}
  % \usepackage{isomath}
  \else % if luatex or xetex
  % \usepackage{fourier}
  % \usepackage{mathpazo}
  % \usepackage{cmbright}
  % \usepackage{isomath}
  \usepackage{fontspec}
  \usepackage{unicode-math}
  \defaultfontfeatures{Scale=MatchLowercase}
  \defaultfontfeatures[\rmfamily]{Ligatures=TeX,Scale=1}
  % \setmainfont{Optima}
  % \setsansfont{Optima}
  % \usepackage{unicode-math}
  % \setmathfont{Optima}
  % \setmathfont{XITS Math}
\fi
% Use upquote if available, for straight quotes in verbatim environments
\IfFileExists{upquote.sty}{\usepackage{upquote}}{}
\IfFileExists{microtype.sty}{% use microtype if available
  \usepackage[]{microtype}
  \UseMicrotypeSet[protrusion]{basicmath} % disable protrusion for tt fonts
}{}
\makeatletter
\@ifundefined{KOMAClassName}{% if non-KOMA class
  \IfFileExists{parskip.sty}{%
    \usepackage{parskip}
  }{% else
    \setlength{\parindent}{0pt}
    \setlength{\parskip}{6pt plus 2pt minus 1pt}}
}{% if KOMA class
  \KOMAoptions{parskip=half}}
\makeatother
\usepackage{xcolor}
\IfFileExists{xurl.sty}{\usepackage{xurl}}{} % add URL line breaks if available
\IfFileExists{bookmark.sty}{\usepackage{bookmark}}{\usepackage{hyperref}}
\hypersetup{
  pdftitle={COMP810 Data Warehousing and Big Data Assessment 2 Data Warehousing Project},
  pdfauthor={Stone Fang (Student ID: 19049045)},
  hidelinks,
  pdfcreator={LaTeX via pandoc}}
\urlstyle{same} % disable monospaced font for URLs
\usepackage[margin=25mm]{geometry}
\usepackage{graphicx,grffile}
\makeatletter
\def\maxwidth{\ifdim\Gin@nat@width>\linewidth\linewidth\else\Gin@nat@width\fi}
\def\maxheight{\ifdim\Gin@nat@height>\textheight\textheight\else\Gin@nat@height\fi}
\makeatother
% Scale images if necessary, so that they will not overflow the page
% margins by default, and it is still possible to overwrite the defaults
% using explicit options in \includegraphics[width, height, ...]{}
\setkeys{Gin}{width=\maxwidth,height=\maxheight,keepaspectratio}
% Set default figure placement to htbp
\makeatletter
\def\fps@figure{htbp}
\makeatother
\setlength{\emergencystretch}{3em} % prevent overfull lines
\providecommand{\tightlist}{%
  \setlength{\itemsep}{0pt}\setlength{\parskip}{0pt}}
\setcounter{secnumdepth}{5}
\pagestyle{empty}

% \usepackage{ifthen}
% Using fancy headers and footers
\usepackage{fancyhdr}
\pagestyle{fancy}
\fancyhead{} % clear all header fields
\fancyhead[L]{COMP810 Data Warehousing and Big Data Assessment 2}
% \fancyhead[CO,CE]{\ifthenelse{\value{page}=1}{first page}
% {COMP810 Data Warehousing and Big Data Assessment 2}}
\fancyfoot{} % clear all footer fields
\fancyfoot[LO,RE]{Stone Fang (19049045)}
\fancyfoot[LE,RO]{\thepage}
\renewcommand{\headrulewidth}{0.4pt}
\renewcommand{\footrulewidth}{0.4pt}

\renewcommand*{\thefootnote}{ [\arabic{footnote}]}

\fancypagestyle{plain}{\pagestyle{fancy}}
\usepackage[]{biblatex}
% redefine the booktitle macro or apacase directive to preserve the source casing for the booktitle field in the @inproceedings
\renewbibmacro*{booktitle}{\printfield{booktitle}}
% \DeclareFieldFormat{apacase}{%
%   \ifboolexpr{ test {\ifentrytype{inproceedings}}
%     and ( test {\ifcurrentfield{booktitle}}
%           or test {\ifcurrentfield{booksubtitle}} ) }
%     {#1}{\MakeSentenceCase*{#1}}}
\addbibresource{ass1.bib}

\title{COMP810 Data Warehousing and Big Data\\
Assessment 2 Data Warehousing Project}
\usepackage{etoolbox}
\makeatletter
\providecommand{\subtitle}[1]{% add subtitle to \maketitle
  \apptocmd{\@title}{\par {\large #1 \par}}{}{}
}
\makeatother
\subtitle{Building and Analysing a DW for NatureFresh Stores in NZ}
\author{Stone Fang (Student ID: 19049045)}
\date{}

\begin{document}
\maketitle

\setstretch{1.25}
\hypertarget{project-overview}{%
\section{Project overview}\label{project-overview}}

The goal of this project is to create a Data Warehouse (DW) for the
sales analysis of NatureFresh, one of the largest fresh food market
chains in New Zealand. Analysis of sales and customer shopping
behaviours can give NatureFresh in-depth insight of the market, so they
can improve their selling strategies accordingly.

The original available data are customer transactions and product
information. The transaction data contains records of customer buying,
including who (customer) bought what (product), when (date),
where(store) and how many was bought (quantity). The product data
contains information for each product, including supplier and price.

However, the format of original data doesn't fit into the requirement of
OLAP, so first we need transform the data into other formats for better
querying.

The major content of this project contains:

\begin{itemize}
\tightlist
\item
  Design and implement the star-schema for sales DW, i.e.~fact \&
  dimension tables
\item
  Fill DW by ETL process. Specifically, do Index Nested Loop Join (INLJ)
  on transactions and master data, transform and load data into fact \&
  dimension tables.
\item
  Execute queries on DW
\end{itemize}

All the operations above are implemented in SQL.

\hypertarget{schema-for-dw}{%
\section{Schema for DW}\label{schema-for-dw}}

According to the original data, the DW will consist of one fact table
\emph{Sales} and five dimension tables \emph{Product}, \emph{Supplier},
\emph{Customer}, \emph{Store}, and \emph{Date}, as shown in Figure
\ref{fig:overall}. The SQL code to create all tables are in file
\emph{createDW.sql}.

\begin{figure}[htbp]
  \centering
  % \sffamily
  { %\fontfamily{phv}\selectfont
  \fontsize{9}{10}\selectfont
  % \fontsize{7}{7}\selectfont
  \def\svgwidth{0.9\columnwidth}
    \resizebox{0.9\textwidth}{!}{\input{star-schema.pdf_tex}}
  }
  \caption{Star Schema of NatureFresh Sales}
  \label{fig:overall}
\end{figure}

\hypertarget{fact-table}{%
\subsection{Fact Table}\label{fact-table}}

Apparently the fact table should have foreign keys corresponding to all
five dimension tables, and the quantity of item sold. There are two
decisions have been make for primary key and amount of money in sales.

\textbf{Primary key} of fact table can be a combination of all foreign
keys. However, there could be a concern to have more than one
transactions for the same values on all five dimensions. A quick
analysis shows that such situation does exists, though the possibility
is low. In other words, a customer may buy one product multiple times at
one store in one day. There are two options to solve this problem. One
is summing up the quantities of multiple transactions, resulting in only
one record for the same combination of dimension values. The other is
keep multiple transactions while use a separated ID field as the primary
key of \emph{Sales} fact table. In this project, the latter solution is
preferred because this approach can keep the original granularity of
transactions, thus contains more information. Also, the possibility of
multiple transactions for one combination of dimensions is low, so there
would not be significant overhead in terms of memory and storage.

\textbf{Price/Amount} is another concerning field. In the original data,
\emph{price} is stored in master data table as a property of product, so
it is natural to make it an attribute of product dimension. However,
this design has a shortcoming when price changes as it always does. If
the price of a product changes, we can't simply modify the value in
\emph{Product} dimension table otherwise the result on sales before that
change will be incorrect. Therefore, in this project price information
is kept in \emph{Sales} table. Since the amount of money in sales is a
more frequent used number, we add to fact table an \emph{amount} filed
which is calculated by \(\mathit{quantity}\times \mathit{price}\). In
section \ref{discussion} further discussions will be provided on this
issue.

\hypertarget{dimensions}{%
\subsection{Dimensions}\label{dimensions}}

Details of dimension tables can be referred to Figure \ref{fig:overall}.
Most dimensions are as simple as ``ID+name'', while the \emph{Date}
dimension is relatively complicated. First of all, unlike other
dimensions, there is no existing ID for \emph{Date}. In this project, a
string in format of ``YYYYMMDD'' is chosen as the ID for \emph{Date},
rather than an auto-incremental column. The advantage is such ID is more
readable and intuitive, and thus more convenient for partitioning if
required in the future. On the other hand, it would need more storage
space, which, however, is not a big issue providing the cost storage is
quite low nowadays. Second, \emph{Date} dimension contains more
information other than names. In this project, common properties are
calculated, including \emph{year}, \emph{quarter}, \emph{month},
\emph{week}, \emph{day}, \emph{day\_of\_week}. In fact, it can be
extended to more fields such as \emph{is\_public\_holiday}, if some
analysis on holiday is in demand.

\hypertarget{inlj-algorithm}{%
\section{INLJ algorithm}\label{inlj-algorithm}}

Index Nested Loop Join (INLJ) is a table joining algorithm that can be
used for stream data joining. Nested Loop Join takes an outer loop and
an inner loop, each for one table, and output the rows that matches the
conditions, so the time complexity is \(O(N M)\) where \(N\) and \(M\)
are the number of rows of two tables. . However, INLJ only keep the
outer loop and replace the inner loop with an index-based loop up, thus
greatly reduce the time complexity. For example, if the index is
implemented by B-tree, then complexity of lookup is a logarithm of \(M\)
instead of linear which is the case of the inner loop.

This algorithm is implemented in PL/SQL. First a bulk (50 rows as in
this project) of transactions is read into memories. Then all rows in
the bulk are read one after another, and retrieve the information for
current row from master data by \emph{product\_id}. Then all properties
corresponding to current row are transformed to fit the star schema and
then load into the fact and dimension tables. Please refer to file
\emph{INLJ.sql} for the complete implementation.

\hypertarget{olap-queries-results}{%
\section{OLAP Queries Results}\label{olap-queries-results}}

This section summarise the results of required analysis. The SQL
statements for these queries are referred to file \emph{queriesDW.sql}.

\let\oldsubsection\thesubsection
\renewcommand*{\thesubsection}{Question~\arabic{subsection}}

\hypertarget{section}{%
\subsection{}\label{section}}

\begin{quote}
Determine the top 5 products in Dec 2019 in terms of total sales
\end{quote}

Result:

PRODUCT\_NAME TOTAL\_SALES RANK ------------ ----------- ----

\hypertarget{section-1}{%
\subsection{}\label{section-1}}

\begin{quote}
Determine which store produced highest sales in the whole year?
\end{quote}

Result:

STORE\_NAME TOTAL\_SALES RANK ---------- ---------------- ----

\hypertarget{section-2}{%
\subsection{}\label{section-2}}

\begin{quote}
Determine the top 3 products for a month (say, Dec 2019), and for the 2
months before that, in terms of total sales.
\end{quote}

Result:

PRODUCT\_NAME SUM(TOTAL\_SALES) RANK ------------- ---------------- ----

\hypertarget{section-3}{%
\subsection{}\label{section-3}}

\begin{quote}
Create a materialised view called ``STOREANALYSIS'' that presents the
product-wise sales analysis for each store. The results should be
ordered by StoreID and then ProductID.
\end{quote}

Result:

STOREID PRODUCTID SUM(STORE\_TOTAL) -------- ---------- ----------------

\hypertarget{section-4}{%
\subsection{}\label{section-4}}

\begin{quote}
Think about what information can be retrieved from the materialised view
created in Q4 using ROLLUP or CUBE concepts and provide some useful
information of your choice for management.
\end{quote}

\renewcommand*{\thesubsection}{\oldsubsection}

\hypertarget{discussion}{%
\section{Discussion}\label{discussion}}

x

\hypertarget{summary-of-what-was-learnt}{%
\section{Summary of what was learnt}\label{summary-of-what-was-learnt}}

x

\printbibliography

\end{document}

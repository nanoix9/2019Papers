\documentclass[conference]{IEEEtran}
\IEEEoverridecommandlockouts
% The preceding line is only needed to identify funding in the first footnote. If that is unneeded, please comment it out.
\usepackage[colorlinks=true, linkcolor=blue, urlcolor=blue, citecolor=blue, anchorcolor=blue]{hyperref}
% \usepackage{cite}
\usepackage{amsmath,amssymb,amsfonts}
\usepackage{algorithmic}
\usepackage{graphicx}
\usepackage{textcomp}
\usepackage[style=ieee,backend=bibtex]{biblatex}
\bibliography{ass1-tmp.bib}

\def\BibTeX{{\rm B\kern-.05em{\sc i\kern-.025em b}\kern-.08em
    T\kern-.1667em\lower.7ex\hbox{E}\kern-.125emX}}
\begin{document}

\title{XXX\\
{\footnotesize \textsuperscript{*}Note: Sub-titles are not captured in Xplore and
should not be used}
\thanks{Identify applicable funding agency here. If none, delete this.}
}

\author{\IEEEauthorblockN{1\textsuperscript{st} Given Name Surname}
\IEEEauthorblockA{\textit{dept. name of organization (of Aff.)} \\
\textit{name of organization (of Aff.)}\\
City, Country \\
email address}}

\maketitle

\begin{abstract}
This document is a model and instructions for \LaTeX.
This and the IEEEtran.cls file define the components of your paper [title, text, heads, etc.]. *CRITICAL: Do Not Use Symbols, Special Characters, Footnotes, 
or Math in Paper Title or Abstract.
\end{abstract}

\begin{IEEEkeywords}
component, formatting, style, styling, insert
\end{IEEEkeywords}

\begin{quote}
An appropriate title of the report /5
\end{quote}

\begin{quote}
Abstract /10 The abstract should be one paragraph and it should cover
the whole theme of the report.
\end{quote}

\hypertarget{introduction}{%
\section{Introduction}\label{introduction}}

\begin{quote}
/10 The introduction part should introduce the topic and its importance.
It should also include at least one diagram related to the topic. One
page is sufficient for this section.
\end{quote}

\hypertarget{backgroundmotivation}{%
\section{Background/motivation}\label{backgroundmotivation}}

\begin{quote}
/10 Background is important to understand the topic in depth while
motivation presents the importance of the topic statement of objectives;
two themes identified for the report should be clearly stated. One page
is sufficient for this section.
\end{quote}

\hypertarget{related-workliterature-review}{%
\section{Related work/literature
review}\label{related-workliterature-review}}

\begin{quote}
/15 Related work should comprise the review of current state of
knowledge relevant to the topic. Comparison and contrasting between
different authors/approaches should be a clear. Page length is 1 to 2.
\end{quote}

\hypertarget{discussionopinion}{%
\section{Discussion/Opinion}\label{discussionopinion}}

\begin{quote}
/20 This is an important section in which the student will criticise the
existing work and will present his/her own opinion about how to improve
it further? This section should reflect some research insight developed
by the student.
\end{quote}

hahaha tdwtd htdwt \autocite{Nobody06}.

\autocite{Nobody06} is a study. \autocite{Ludwig1996}

goog \autocite{Ludwig1996}

\hypertarget{conclusion}{%
\section{Conclusion}\label{conclusion}}

\begin{quote}
/10 In this section student will draw conclusions on the given topic. In
other words, it is a brief summary about work presented in the research
report. Future issues /10 This section should discuss at least two
future issues on the topic.
\end{quote}

\hypertarget{references}{%
\section{References}\label{references}}

\begin{quote}
/10 All references should be of peer-reviewed journal and conferences.
They must be clickable in the document. Each report should include at
least 6 peer-reviewed references.
\end{quote}

\href{http://google.com}{google}

% \bibliography{ass1}{}
% \bibliographystyle{IEEEtran}

\printbibliography

\end{document}

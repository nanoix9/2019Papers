\documentclass[conference]{IEEEtran}
\IEEEoverridecommandlockouts
% The preceding line is only needed to identify funding in the first footnote. If that is unneeded, please comment it out.
\usepackage[colorlinks=true, linkcolor=blue, urlcolor=blue, citecolor=blue, anchorcolor=blue]{hyperref}
% \usepackage{cite}
\usepackage{amsmath,amssymb,amsfonts}
\usepackage{algorithmic}
\usepackage{graphicx}
\usepackage{textcomp}
% \usepackage[mathletters]{ucs}
\usepackage[utf8]{inputenc}
\usepackage[T1]{fontenc}
\usepackage[normalem]{ulem}
% Avoid problems with \sout in headers with hyperref
\pdfstringdefDisableCommands{\renewcommand{\sout}{}}
\usepackage[style=ieee,backend=bibtex]{biblatex}
\bibliography{ass1.bib}

\providecommand{\tightlist}{%
  \setlength{\itemsep}{0pt}\setlength{\parskip}{0pt}
}

\def\BibTeX{{\rm B\kern-.05em{\sc i\kern-.025em b}\kern-.08em
    T\kern-.1667em\lower.7ex\hbox{E}\kern-.125emX}}
\begin{document}

\title{(An appropriate title of the report /5)\\
{\footnotesize \textsuperscript{*}Note: Sub-titles are not captured in Xplore and
should not be used}
% \thanks{Identify applicable funding agency here. If none, delete this.}
}

\author{\IEEEauthorblockN{Stone Fang (Student ID: 19049045)}
\IEEEauthorblockA{\textit{Computers and Information Sciences} \\
\textit{Auckland University of Technology}\\
Auckland, New Zealand \\
fnk7060@autuni.ac.nz}}

\maketitle

\begin{abstract}

xxxxxx xxxxxxxxxxxxxxxxx xxxxxx \textgreater{} Abstract /10
\textgreater{} The abstract should be one paragraph and it should cover
the whole theme of the report.

This document is a model and instructions for \LaTeX.
This and the IEEEtran.cls file define the components of your paper [title, text, heads, etc.]. *CRITICAL: Do Not Use Symbols, Special Characters, Footnotes, 
or Math in Paper Title or Abstract.
\end{abstract}

\begin{IEEEkeywords}
Big Data, scalability
\end{IEEEkeywords}

\hypertarget{introduction}{%
\section{Introduction}\label{introduction}}

\begin{quote}
/10 The introduction part should introduce the topic and its importance.
It should also include at least one diagram related to the topic. One
page is sufficient for this section.
\end{quote}

\hypertarget{what-is-big-data.}{%
\subsection{\texorpdfstring{\sout{what is Big
Data.}}{what is Big Data.}}\label{what-is-big-data.}}

In recent years, exponential increasing volumes of data have been
generated by a variety of sources such as e-businesses, communications,
mobile phones, social media sites, web servers, sensor networks,
cameras, banks, stock markets, and so on
\autocite{OUSSOUS2018431,SIVARAJAH2017}. Nowadays, data are being
generated at an unprecedented rate, bringing us into a era of Big Data
or ``data deluge'' \autocite{SIVARAJAH2017,hu2014}. The types of data
vary from text to multimedia including image, audio, and video. The
format can be structured, semi-structured and unstructured. In the real
world, more than 90\% of the overall data generated are unstructured
\autocite{SIVARAJAH2017}.

\hypertarget{value-importance}{%
\subsection{\texorpdfstring{\sout{value,
importance}}{value, importance}}\label{value-importance}}

Big Data has enormous potential values with the expectation of
transforming the humans society, and is consequently regarded as the
``new oil'' by some researchers \autocite{hu2014}. It not only can
brings large amount of revenues to businesses and values to consumers,
but also has great potential applications in a range of industries. For
instance, health or medical data analysis has many benefits including
personalised health service, disease evolution monitoring, and adaptive
public health plans \autocite{OUSSOUS2018431}. Another example is the
real-time analysis of data generated by smart meters, sensors and
control devices on smart grid, which can help in incident detection,
risk identification, and energy consumption forecast
\autocite{OUSSOUS2018431}.

\begin{quote}
``Google, Amazon, Facebook and Twitter gained enormous advantages from
big data methodologies and techniques.'' \autocite{Hewage2018}
\end{quote}

\begin{quote}
``For instance, the opportunities include value creation (Brown, Chui,
\& Manyika, 2011), rich business intelligence for better- informed
business decisions (Chen \& Zhang, 2014), and support in enhancing the
visibility and flexibility of supply chain and resource allocation
(Kumar, Niu, \& Ré, 2013).'' \autocite{SIVARAJAH2017}
\end{quote}

\hypertarget{challenges-especially-on-scalability}{%
\subsection{\texorpdfstring{\sout{challenges, especially on
scalability}}{challenges, especially on scalability}}\label{challenges-especially-on-scalability}}

Before Big Data evolution, it is difficult to store, manage or analyse
data sets in large volumes because of the limited store capacity and
lack of scalability, flexibility and performance in traditional
technologies \autocite{OUSSOUS2018431}. Relational database management
system (RDBMS), the main technology in traditional data management, does
not fit the requirements in Big Data analysis. The reasons of this
mismatch are twofold, including:

\begin{itemize}
\tightlist
\item
  Data structure: RDBMS only supports structured data, while has little
  capability in storage and analysis of semi-structured and unstructured
  data \autocite{hu2014}.
\item
  Scalability: RDBMS only scales up at high costs of hardware, and is
  very difficult to scale out, which makes it incapable in continuously
  growing data scenarios \autocite{hu2014,SIVARAJAH2017}.
\end{itemize}

\hypertarget{popularity}{%
\subsection{\texorpdfstring{\sout{popularity}}{popularity}}\label{popularity}}

As a result of the big values and big challenges inside Big Data,
interests from both academia and industry are dramatically increasing in
recent years. A wide range of issues have been studied at different
levels including data storage, cleaning, analysis, visualisation, and so
on, some of them still open to research \autocite{OUSSOUS2018431}. In
the industry, many companies have their own Big Data platforms, for
example, Google's large data storage Google File System(GFS) and cloud
based data management system Fusion Table \autocite{Hewage2018}. Many
Big Data systems and platforms including open-source ones have been
being developed, for instance, NoSQL Databases, BigQuery, MapReduce,
Hadoop, HiveQL, Spark, to mention but a few
\autocite{Hewage2018,SIVARAJAH2017,hu2014}. Some projects have also been
launched by governments of countries such as USA and Japan to catch Big
Data opportunities \autocite{OUSSOUS2018431}.

\hypertarget{backgroundmotivation}{%
\section{Background/motivation}\label{backgroundmotivation}}

\begin{quote}
/10 Background is important to understand the topic in depth while
motivation presents the importance of the topic statement of objectives;
two themes identified for the report should be clearly stated. One page
is sufficient for this section.
\end{quote}

\hypertarget{in-depth-understanding-of-big-data}{%
\subsection{\texorpdfstring{\sout{in depth understanding of Big
Data}}{in depth understanding of Big Data}}\label{in-depth-understanding-of-big-data}}

The main characteristics of Big Data are described as three Vs, namely
Volume, Velocity and Variety \autocite{OUSSOUS2018431,hu2014}. First of
all, the large volume of data is an essential difference between Big
Data and traditional data \autocite{hu2014}. Second, the velocity at
which the data are being generated implies that the processing and
analysis of datasets should be carried out at a comparable rate to the
data production \autocite{hu2014}. Third, Big Data are produced in
various format including both text and multimedia from various data
sources, resulting in high heterogeneity and diversity
\autocite{OUSSOUS2018431,Pouyanfar:2018}.

According to a well-accepted system engineering methodology in industry,
the Big Data value chain is decomposed into four consecutive stages
\autocite{hu2014}:

\begin{itemize}
\tightlist
\item
  \textbf{Data generation} refers to the processes that data are
  generated from various sources.
\item
  \textbf{Data acquisition} focuses on the obtaining and collection of
  data.
\item
  \textbf{Data storage} concerns the persistent data storage and
  effective data management.
\item
  \textbf{Data analytics} is the stage concerning the extraction of
  value from data by exploring, transforming, modelling and visualising
  data with analytical tools.
\end{itemize}

\hypertarget{motivation-importance-of-big-data}{%
\subsection{\texorpdfstring{\sout{motivation \& importance of Big
Data}}{motivation \& importance of Big Data}}\label{motivation-importance-of-big-data}}

\hypertarget{challenges-of-opportunities-of-big-data}{%
\subsection{\texorpdfstring{\sout{challenges of opportunities of Big
Data}}{challenges of opportunities of Big Data}}\label{challenges-of-opportunities-of-big-data}}

The mismatch between the requirements of Big Data and existing data
management hardware and software platforms raises many challenges to
both industry and research community. Many researches and practices,
especially relating to the scalability, have been conducted among the
four phases, including:

\begin{itemize}
\tightlist
\item
  Network architectures and protocols with high throughput, low latency
  and optimal energy consumption for large-scale data transmission
  \autocite{hu2014}
\item
  Scalable data cleaning, aggregation and duplication removal method for
  huge dataset at reasonable speed but still with acceptable accuracy.
  It is essential for big data quality and reliability
  \autocite{hu2014,OUSSOUS2018431}
\item
  Infrastructures, file systems and database technologies for
  distributed and scalable data storage. More specific issues include
  data partitioning and replication scheme, scalable data indexing and
  query, CAP option (consistency, availability and partition tolerance),
  concurrency control mechanism, parallel and distributed programming
  model \autocite{hu2014,Gupta2016}.
\item
  Scalable machine learning on large dataset, including deep learning on
  large dataset, online (or stream) learning, parallel reinforcement
  learning, computation framework for machine learning, and so on
  \autocite{OUSSOUS2018431,Gupta2016}
\item
  Real-time or near real-time analysis of large increasing volume of
  data \autocite{OUSSOUS2018431}
\item
  Imbalanced Big Data analysis \autocite{OUSSOUS2018431}
\end{itemize}

\hypertarget{themes-to-dive-in}{%
\subsection{\texorpdfstring{\sout{themes to dive
in}}{themes to dive in}}\label{themes-to-dive-in}}

\hypertarget{related-workliterature-review}{%
\section{Related work/literature
review}\label{related-workliterature-review}}

\begin{quote}
/15 Related work should comprise the review of current state of
knowledge relevant to the topic. Comparison and contrasting between
different authors/approaches should be a clear. Page length is 1 to 2.
\end{quote}

reviews of surveys on Big Data

reviews of publications on the themes

\hypertarget{discussionopinion}{%
\section{Discussion/Opinion}\label{discussionopinion}}

\begin{quote}
/20 This is an important section in which the student will criticise the
existing work and will present his/her own opinion about how to improve
it further? This section should reflect some research insight developed
by the student.
\end{quote}

opinion on theme 1

opinion on theme 2

\ldots{}

\hypertarget{conclusion}{%
\section{Conclusion}\label{conclusion}}

\begin{quote}
/10 In this section student will draw conclusions on the given topic. In
other words, it is a brief summary about work presented in the research
report.
\end{quote}

\hypertarget{future-issues}{%
\section{Future issues}\label{future-issues}}

/10 This section should discuss at least two future issues on the topic.

\hypertarget{references}{%
\section{References}\label{references}}

\begin{quote}
/10 All references should be of peer-reviewed journal and conferences.
They must be clickable in the document. Each report should include at
least 6 peer-reviewed references.
\end{quote}

% \bibliography{ass1}{}
% \bibliographystyle{IEEEtran}

\printbibliography

\end{document}
